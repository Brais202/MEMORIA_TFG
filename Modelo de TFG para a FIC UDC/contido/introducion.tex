\chapter{Introduction}
\label{chap:introduction}
Access Control is about controlling who can access a resource, either a physical building, a digital computer system or sensitive data. In the context of physical facilities, such as corporate offices, data centers, laboratories or manufacturing plants, access control systems integrate hardware (e.g., electromagnetic locks, microcontrollers or NFC/RFID readers) and centralized management software to enforce entry rules and audit log events into their solutions. By defining roles, schedules and zones of authorization, these solutions ensure that only properly credentialed individuals enter the right areas at the right times, reducing unauthorized entry, tailgating and insider threats \cite{Ref1}.

Over the past decade, advancements in wireless communication standards have driven the adoption of contactless technologies in security‑critical environments. Among these, Near Field Communication (NFC) stands out for its global standardization (ISO/IEC 14443)~\cite{Ref23}, interoperability with billions of mobile devices, and inherently short communication range, which mitigates relay‑attack risks \cite{Ref73}.

In the domain of access control, NFC has proven particularly effective, offering a balance between usability and security~\cite{Ref72}. Its contactless nature allows for rapid authentication at entry points, such as door access systems in offices or public transportation fare systems, where speed and convenience are critical. Also, NFC systems can be easily integrated with smartphones, enabling secure mobile payments and access control through applications, while also supporting robust encryption methods to safeguard transmitted data. It operates by leveraging tags or cards embedded with unique identifiers and cryptographic keys. These credentials are read by NFC-enabled devices to verify access rights in real time. Despite its advantages, NFC-based access control faces significant challenges, including the risks of cloning, unauthorized interception, and tampering~\cite{Ref2}.

However, most of the systems still employ basic cards such as MIFARE Classic~\cite{Ref28}, whose security is based on a set of static, reversible keys that can be extracted with low-cost hardware tools. These cards have been subject to multiple practical attacks after the release of their proprietary algorithms more than a decade ago~\cite{Ref79}. As a result, they remain vulnerable to spoofing and data manipulation, forcing today's migration to more robust technologies (e.g., NTAG 424 DNA with AES-128 and HMAC-SHA256) to ensure integrity and confidentiality in critical access control systems.

The growing demand for NFC physical access control solutions has been met with a lack of open and affordable alternatives~\cite{Ref3}. Most traditional access control systems rely on proprietary hardware such as RFID/NFC card readers, central management servers, and closed-source software platforms provided by commercial vendors. These solutions come from private companies such as HID Global~\cite{Ref4}, leader in the sector, whose complete implementations cost over 5,000\,€, and others such as Suprema~\cite{Ref5} or Honeywell~\cite{Ref6}. The pricing of these solutions is usually not affordable for Small and Medium-sized Enterprises (SMEs) and private users. In addition, these platforms tend to be closed and not very customizable. For example, HID SEOS\textregistered~\cite{Ref7}, despite its cryptographic robustness and cross-platform support, does not allow credentials to be generated or integrated into third-party environments without proprietary tools and licenses. While many vendors provide turnkey NFC‑based door controllers, these typically lock customers into proprietary ecosystems with high per‑unit costs. Moreover, closed APIs and opaque credential‑management workflows hinder integration with custom identity‑management platforms. This approach makes the system expensive and limits its flexibility~\cite{Ref74}.

This degree thesis project develops a modular, open-source, and cost-effective physical access control system using NFC. Motivated by the lack of affordable transparent solutions and the potential of NFC technology for securing physical facility entry, the developed system uses NFC cards, readers, and a web server to securely manage electronic locks and record entry events in real time. System administrators are able to configure user profiles, apply time-based permissions and centrally view a detailed access history through a web dashboard. In addition, the solution incorporates advanced cryptographic mechanisms (e.g., NTAG424, TLS or AES-128) to mitigate threats such as card cloning, unauthorized interception and data tampering. Furthermore, regarding the physical layer, the system leverages a microcontroller integrated with dedicated hardware modules, specifically designed for seamless interaction with NFC tags and secure communication with the web server.

\clearpage
\section{Objectives}
\label{sec:objectives}


The main objective of this work is to develop a modular, open source, cost-effective access control system using NFC technology to securely manage electronic locks and record entry events in real time.

The specific objectives are the following:
\begin{itemize}
	\item Analyze the technologies that are being used in the field of access control and identify the main security issues.
	\item Analyze the system requirements, design the system architecture, the protocol operation and the key management system.
	\item Develop and validate a working prototype using an iterative design approach, in which essential hardware components are progressively integrated until a complete access control system is achieved.
	\item Secure the communication between the different modules and implement security measures to avoid the main threats faced by NFC-based access control systems (e.g., data leaks, card cloning or impersonation).
	\item Develop a role-based web-based management system that allows the assignment and configuration of time-dependent access permissions, enabling administrators to precisely define authorized entry intervals for each user profile.
	\item Integrate a real-time alert mechanism that notifies administrators of access attempts outside predefined schedules, thus ensuring effective control and increased security in the access control system.
\end{itemize}

